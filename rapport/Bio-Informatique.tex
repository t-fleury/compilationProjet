\documentclass[12pt, openany]{report}
\usepackage[T1]{fontenc}
\usepackage[a4paper,left=2cm,right=2cm,top=2cm,bottom=2cm]{geometry}
\usepackage[frenchb]{babel}
\usepackage{libertine}
\usepackage{hyperref}
\usepackage[pdftex]{graphicx}

%Ajouter des r�gles
\newcommand{\HRule}{\rule{\linewidth}{0.5mm}}

				
\begin{document}

	% Page de pr�sentation
	\begin{titlepage}
	\begin{center}
		 \includegraphics[scale=0.45]{universite-Bordeaux.jpg}~\\[1.5cm]
		\textsc{\LARGE Licence 3 Informatique}\\[2cm]
		
		\textsc{\Large Rapport en Compilation}\\[1.5cm]
		
		% Titre
		\HRule \\[0.4cm]
		{  \huge{\bfseries Projet Compilation}\\[0.4cm] }
		\HRule \\[2cm]
		
		% Auteurs
		\textsc{GUILBAULT Maxime, BARRIERE Antoine, FLEURY Thomas, MAYOLINI Maxime}\\[0.4cm]
		
		% Bas de page
    		{\large \today}
	\end{center}
	\end{titlepage}
	
	\tableofcontents~\\[18cm]

% Remerciements
\begin{center}
\end{center}~\\[1.5cm]
\section*{Contexte}~\\[0.5cm]

\large{Nous sommes un groupe de 4 �tudiants en Licence 3 d?informatique de l?Universit� de Bordeaux compos�s de MAYOLINI Maxime, FLEURY Thomas, BARRIERE Antoine et GUILBAULT Maxime. Ce document correspond au compte-rendu du projet que nous avons d� r�aliser dans le courant de notre sixi�me semestre en Compilation. Le projet consiste � analyser le langage Pseudo-Pascal, � l'interpr�ter, le traduire en C3A et � �crire un interpr�te de C3A. 

Pour cela, nous utilisons le lex et le bison comme langage afin de pouvoir analyser la grammaire et ainsi traduire afin de pouvoir faire un compilateur de Pseudo-Pascal en C3A} \\[8cm]

\addcontentsline{toc}{section}{Contexte}



% Premi�re Partie 
\chapter{Introduction}
Il est dans notre nature de vouloir comprendre le monde qui nous entoure. Dans notre �volution, l'Homme cherche � d�chiffrer ce monde plein de ressources et de myst�res. Pour ce faire, il cr�a les sciences ('scientia' qui en Latin signifie "connaissance") qui refl�tent nos �tudes sur cette terre.  Plus nous �voluons, plus nous acquerrons des connaissances qui nous approche de la r�alit�. Mais cet univers est encore plein de myst�res.

\addcontentsline{toc}{section}{D�finitions}

\end{document}  



